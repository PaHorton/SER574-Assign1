\documentclass[sigplan,screen]{acmart}

\def\BibTeX{{\rm B\kern-.05em{\sc i\kern-.025em b}\kern-.08emT\kern-.1667em\lower.7ex\hbox{E}\kern-.125emX}}

\copyrightyear{2019}
\acmYear{2019}
\setcopyright{acmlicensed}
\acmConference[Mesa '18]{Mesa '18: SER 574}{Jan 20, 2019}{Mesa, AZ}
\acmDOI{11.2222/3333333.4444444}
\acmISBN{111-2-333-4444-1/20/19}



\begin{document}

\title{Designing an Agile Collaboration: How to Succeed in Cross-Team Interactions}

\author{Paul Horton}
\email{pahorton@asu.edu}
\affiliation{%
  \institution{Arizona State University}
  \city{Mesa}
  \state{Arizona}
  \postcode{85212}
}

\author{Ruby Zhao}
\email{qrzhao@asu.edu}
\affiliation{%
  \institution{Arizona State University}
  \city{Mesa}
  \state{Arizona}
  \postcode{85212}
}

\begin{abstract}
Agile models within the field of software engineering are widely used to facilitate flexibility and ease of communication in order to solve dynamic engineering challenges.
For agile teams to succeed on a complex project, they must be able to work not only with their own members but also with other teams.
The key to collaboration in an agile setting is strong architectural planning that allows for multiple teams to contribute simultaneously.
By creating a valid architecture that allows for collaborative programming, all participants in the development are given immediate feedback.
This ultimately leads to better software products developed in a more efficient manner.
\end{abstract}

\keywords{agile, teamwork, design, architecture, project management, software engineering}

\maketitle

\section{Introduction}
Agility is currently the byword of the software engineering field.
Software engineers must respond to customer requirements with speed and flexibility, and for many projects, development teams have turned to Agile methodology as the answer to these demands.
A good agile team is one that communicates, but is also responsive to, changing requirements to and from other teams.
Understanding the communication techniques used on best practice agile teams reveals optimal development strategies for future projects.
This work primarily focuses on cross-team interaction methodology for agile development.


The section "Important Aspects" will discuss which qualities allow teams to work together successfully.
This section lists actionable qualities of a healthy team dynamic that can be incorporated into any team making use of agile processes.
The following section, "Design and Architecture", discusses the role of design and architecture in cross-team communication efforts.
This section highlights the importance of architectural design as a tool to assist cross-team development as well as how design planning and agile can coexist.

\section{Agile Collaboration}

\subsection{Important Aspects}

\subsection{Design and Architecture}

\section{Conclusion}



\bibliographystyle{ACM-Reference-Format}
\bibliography{sample-base}


\end{document}
