% !TEX root = ../Horton_Zhao_Assignment1.tex
\subsection{Design and Architecture}
Architectural planning is a vital part of successful cross-team agile development despite the perceived conflict between the two practices.
A key aspect of agile is high adaptability to change over following strict plans \cite{AgileMani}.
This seems to directly contrast the traditional viewpoint for architecture which provides the "floor plans" as framework directly tied to the requirements of the project \cite{perry1992foundations}.
At first glance, the two are at odds on the fundamental level with agilists concerned with Big Up Front Design (BUFD) and You Ain't Gonna Need It (YAGNI) features and architecturalists seeing the agile methodology as amature \cite{kruchten2010software}.
The problems in this dynamic arise when an architectural team design a team for an implementation team to create.
When changes arise in requirements, "the architecture team do not always fully understand the repercussions their design changes will have on existing and new product features," \cite{isham2008agile}.
By integrating the architecture and implementation teams into agile scrum teams, all developers have a stake in the final product and understand the impacts of changes requirements \cite{isham2008agile}.


The key success in this integration is the sprintable nature of scrum.
With up-front design that decomposes the architecture into sections with architecturally significant boundaries, teams are able to distribute the systems and implement in sprints \cite{madison2010agile}.
Newly designed parts of the architecture can be rolled out in parallel to existing architecture to build up the planned design \cite{isham2008agile}.
By producing the parts in increments, developers are able to "learn and adjust along the way," \cite{isham2008agile}.
To account for change, the architecture can be designed such that a range of options are considered so that modifications can be made as the system is being built \cite{madison2010agile}.
These architectural choices place collaboration at the forefront of the design and ensure that communication between teams must be maintained in order for efficient development.
