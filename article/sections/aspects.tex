% !TEX root = ../Horton_Zhao_Assignment1.tex
\subsection{Communication Across Agile Teams}

Agile methodology prioritizes interaction between individuals.
While this principle is certainly essential to interactions with the customer, it is no less important for agile teams to actively seek out interactions with other teams as well.
Yang et al\cite{YANG} have shown that teamwork and cohesiveness exert a significant influence upon project success.
A lack of cross-team communication can result in conflicts of interest, knowledge silos, and misunderstandings that negatively impact the status of the project.
With a large project, workers must be split into multiple agile teams to address the scope of the work, but the more teams exist, the more issues related to team cooperation are magnified.
Therefore, agile teams must consider what elements of cross-team collaboration are important to ensure the overall success of the project.

The agile principles enumerated in the Agile Manifesto include statements that emphasize the importance of interpersonal relations. Such principles include:
\begin{itemize}
\item "Business people and developers must work together daily throughout the project."
\item "Build projects around motivated individuals. Give them the environment and support they need, and trust them to get the job done."
\item "The most efficient and effective method of conveying information to and within a development team is face-to-face conversation." \cite{AgileMani}
\end{itemize}

These principles prescribe a regular, encouraging, and personal environment in which to nurture collaboration.
The successful agile team should communicate with other teams on a consistent basis, ideally in person.
Team managers should give team members the resources they need to succeed, including resources from other teams.

Another important component for cross-team collaboration is to ensure that team members are invested in communication.
For instance, Crocker et al\cite{HBR} performed an extensive survey through interviews and other methods that showed that amongst multiple teams forming a cohesive agile network, the following are important: "1) managing the center of the network, 2) engaging the fringe, 3) bridging select silos, and 4) leveraging boundary spanners."
Summed up, these four points all focus on one key idea: the importance of managing interactions between key players and experts to allow for successful communication of ideas and issues.
Agile teams must not take communication for granted, but rather must take active ownership of their cross-team participation.
Without an institution-wide, systematic effort to reach out to others, teams run the risk of individual cooperation falling to the wayside and important knowledge being lost.

It is important to keep in mind, however, that different teams are separate entities.
Lee et al\cite{LEE} found in a study that team autonomy encourages creativity and efficiency within the team, but team diversity allows for better problem-solving and distribution of knowledge.
This concept may also be extended to diversity across teams.
Too much collaboration can lead to issues between teams, especially if one team sacrifices their own autonomy in order to continuously assist other teams.
Through ample communication, balance can be achieved with sufficient discussion of task priorities, both within and between teams.
