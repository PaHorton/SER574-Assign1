% !TEX root = ../Horton_Zhao_Assignment1.tex
\section{Conclusion}
In conclusion, cross-team agile developement requires ample communication to ensure the success of the project.
By utilizing the encouraging and personal environment of agile, teams are able to consistently communicate with each other such that key aspects of the project are addressed by experts across teams.
Giving teams the freedom to reach out to other teams when they need to communicate their issues ensures knowledge is shared across the institution.
This collaboration can be encouraged by re-integrating architecture and implementation teams into diverse teams.
While agile and architecture may seem dichotomic, careful architectural planning ensures that the agile philosophy is maintained.
With an emphasis on changability, agile architecture focuses on smaller, iteratable design that provides room to adjust in the future.
Development is done in iterations that produce bits of the architecture in parallel that are integrated together to form the larger structure.
